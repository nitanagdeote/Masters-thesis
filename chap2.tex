\chapter{Error Analysis for the Green's Function and the expansion Coefficient }
\label{intro1}
\section{Introduction}
In this chapter, we will develop the 
Single Layer Fast Multipole Method (SLFMM)
and its application to the solution of 
the Helmholtz equation.
In this context the Fast Multipole Method (FMM) has found application in the numerical analysis of 
problems arising in the fields of electromagnetics, acoustics, astronomy, 
molecular dynamics and fluid mechanics\cite{beatson1997short,coifman1993fast,anderson1992implementation}. 
 
The  FMM is used to evaluate an unknown potential function at a given spatial location that results from the interaction of a large number of source
particles \cite{chartier2010regular}. 
The Green's function $ g( \vec{r},\vec{r}_0 )$ is used to evaluate the influence of each 
source at $\vec{r}_0$ on the observed potential at $\vec{r}$.
Given $N$ source particles and $N_o$ observation points, 
a direct calculation of the potential at all the prescribed 
observation points can be evaluated as the matrix-vector multiplication.
\begin{equation}
p(\vec{r}) = \sum_{j=1}^N  g( \vec{r},\vec{r}_0 )s_j \nonumber
\end{equation}
In such a case the evaluation will require $O( N_o N )$ complex multiplications. 

\section{Multipole expansion}\label{multiEx1}
The multipole expansion relies heavily on the separability of the
kernel function $g(x,y)$. In Chapter 2 separable expressions for the Green's function
have been given for the three-dimensional Helmholtz equation case.
The technical elements required for exploiting the aforementioned result will be given here using a one dimensional model.
Consider the integral 
\begin{equation}\label{first2chap}
p(x)=\int_a^b g(x,y)~s(y)dy
\end{equation} 
where $s(y)$ is the source distribution function.
The Green's function $g(x,y)$ depends on two variables the 
observation point $x$ and the source point $y$. 
The objective is to evaluate $p(x)$.
Let the observation point $x_i$ is drawn 
from the set
$x = \{x_1,x_2, ...,x_{N_o}\}$ and the source point 
is drawn for $y = \{y_1,y_2,...,y_N\} $, then
\begin{equation}
\label{chap3/0.21}
p(x_i)=\sum_{j=1}^{N} g(x_i,y_j)~s(y_j)~w( y_j)
\end{equation} 
for $i=\{1,...,N_o\}$ and
$w( y_j )$ is the quadrature weight at $y_j$.
Solving for $p(x_i)$ using kernel $g(x,y)$ leads to $O(N^2)$ operations.
In such case one will obtain $p$ using $N_o N$ multiplications and $N_o (N-1)$ additions.

To reduce the number of operations, the strategy is to 
represent $g(x,y)$ as a product of a function in  $x$ and a function
in  $y$. 
\begin{equation}
\label{chap3/0.1}
g(x,y)= \sum_{l=0}^{M} f_l(x)q_l(y)
\end{equation} 
 where $f_l(x)$ and $g_l(y)$ are functions depending on $x$ and $y$ respectively. 
Calculating  the integral requires discretization of the integrand. Discretizing    the  function $s(y)$ as follows.
\begin{equation}
\label{chap3/0.3}
s(y) = \sum_{j=1}^{N} \delta(y- y_j)s(y)w(y)
\end{equation} 
Substituting Eqn (\ref{chap3/0.3})  into Eqn (\ref{first2chap})$~$ yields
\begin{equation}
\label{chap3/0.10}
\begin{split}
p(x)= &\int_{a}^{b} g(x,y)~\sum_{j=1}^{N} \delta(y-y_j)s(y)w(y)dy \\
= & \sum_{j=1}^{N} g(x,y_j)~  s(y_j)w(y_j)
\end{split}
\end{equation} 
kernel $g(x,y_j)$ as
\begin{equation}
\label{chap3/0.5}
g(x,y_j)= \sum_{l=0}^{M} f_l(x)q_l(y_j)
\end{equation} 
Substituting Eqn (\ref{chap3/0.5}) into Eqn (\ref{chap3/0.10}) yields
\begin{equation}
\label{chap3/0.9}
\begin{split}
p(x)= &\sum_{j=1}^{N}\sum_{l=0}^{M} f_l(x)q_l(y_j) ~  w(y_j) s(y_j)
\end{split}
\end{equation} 
Grouping the terms which are dependent on $y_j$ only
\begin{equation}
\label{chap3/0.11}
a_{l}= \sum_{j=1}^{N}q_l(y_j)~  w(y_j)~ s(y_j)
\end{equation} 
 Substituting $a_{l}$ in Eqn (\ref{chap3/0.9}) gives
\begin{equation}
\label{chap3/0.13}
p(x)=  \sum_{l=0}^{M} a_{l}f_{l}(x)
\end{equation} 
where $a_l$ is a constant.  
The number of operations required to calculate each 
$a_l$ is $O(N)$ and to calculate $p(x)$ are $O(M+1)$. 
The coefficients $a_{l}$ only needs to be calculated  once which is a non-recurring cost of $NM$ multiplications.
Once the coefficients are evaluated
each $p(x_i)$ requires $(M+1)$ multiplications and $M$ additions.
As a result, $p$ is evaluated at a cost of 
$O(N_o (M+1) + N M)$ multiplications  as compared to
$O(N_o N)$ in the case of a direct calculation.
The critical factor in the calculation is on the
requirement that one can achieve high accuracy  for 
cases where $M$ is much less than $N$.

%----------------------------------------------------------------------------

\section{the FMM Evaluation of the Kirchhoff Integral}\label{chap3-KIF}

In this  section we will outline  how the fast multipole method can be 
applied to the evaluation of the Kirchhoff integral. 
Consider a source distribution which is enclosed in a volume $V$.
The problem at hand is to evaluate the pressure at a spatial point located
external to the volume.
In such a case the pressure can be evaluated by the volume integral
\begin{equation}\label{generalPotInt}
p(\vec{r}) = \int_{V} g(\vec{r},\vec{r}_0)\enskip{s(\vec{r}_0)} \enskip  \textit{dV}_0
\end{equation}   
where  $p(\vec {r})$ is the pressure  at the observation  point $\vec{r}$ 
and   $s(\vec{r}_0)$ represents the source distribution 
in the volume $V$. We will assume that $|\vec{r}|>|\vec{r}_0|$ for all observation points of
interest.

The  source distribution $s(\vec{r}_0)$ is modeled  by its samples located at $N$ points. An element of the set of sample points is 
denoted as
\begin{equation}
\label{chap2/1}
s(\vec{r}_0) = \sum_{j=1}^{N}{\delta(\vec{r}_0-\vec{X}_j)s_j w_j}  
\end{equation}
where, $s_j$ is the amplitude of an individual source located at $\vec{X}_j$
and $w_j$ is the quadrature weight. A schematic of the problem domain is
shown in Fig \ref{source location}. This choice allows one to
replace the volume integral by a summation.
In the figure the origin of the coordinate system is labeled as $O$. 
The task is to find the unknown potential at the observation point which is outside the volume enclosing the sources and  
Eqn (\ref{generalPotInt}) is evaluated numerically.
\begin{figure}
\begin{center}
\includegraphics[scale=0.8]{./chap3.d/fig/chap3_fig3.1.eps}
\caption{Source locations and amplitudes }
\label{source location}
\end{center}
\end{figure}

\begin{equation}
\label{chap2/3}
 p(\vec{r}) = \sum_{j=1}^{N}{s_j w_j}\int_{v} g(\vec{r},\vec{r}_0)\delta(\vec{r}_0-\vec{X}_j)  \enskip  \textit{dV}_0
\end{equation}
Using the integral  property of the  delta function, Eqn (\ref{chap2/3}) can be
rewrittten as 
 \begin{equation}
\label{chap2/5}
p(\vec{r}) = \sum_{j=1}^{N}{ s_j w_j g(\vec{r},\vec{X}_j)}
\end{equation}
To calculate the unknown potential at the observation point in spherical coordinate system, the  Green's function  in  Eqn (\ref{chap2/5}) can be expressed as follows when   $|\vec{r}|>|\vec{X}_j|$, 
\begin{equation}
\label{chap2/7}
g(\vec{r},\vec{X}_j) = i\gamma\sum_{n=0}^{T}\sum_{m=-n}^{n}R_n^{-m}(\vec{X}_j)S_n^{m}({\vec{r})} 
\end{equation}
Substituting Eqn (\ref{chap2/7}) in Eqn (\ref{chap2/5}) yields
\begin{equation}
\label{chap2/9}
p(\vec{r}) =  i\gamma\sum_{n=0}^{T}\sum_{m=-n}^{n}
S_{n}^{m}(\vec{r})
\sum_{j=1}^{N}{ s_j w_j R_n^{-m}(\vec{X}_j)}\hspace{0.5 in}\text{if $ |\vec{r}|>|\vec{X}_{j}|$}
\end{equation}
The summation in the index $j$ can be calculated separately and yields the coefficient $a_{nm}$. 
\begin{equation}
\label{chap2/11}
a_{nm}=\sum_{j=1}^{N}{ s_j w_j R_n^{-m}(\vec{X}_j)}
\end{equation}
As a result Eqn (\ref{chap2/9}) becomes 
\begin{equation}
\label{chap2/12}
p(\vec{r}) =  i\gamma\sum_{n=0}^{T}\sum_{m=-n}^{n}S_{n}^{m}(\vec{r}) a_n^{m} \hspace{0.5 in}\text{if $ |\vec{r}|>|\vec{X}_{j}|$}
\end{equation} 
Using Eqn (\ref{chap2/12}) one can calculate 
the pressure at the observation points outside the volume enclosed 
by source points. The expansion in the  Eqn (\ref{chap2/12}) is referred to as the 
farfield expansion or S-expansion.

\newpage
\section{Multipole Expansion about an Arbitrary Reference Point}
In this section we will examine how one can 
evaluate the multipole expansion about a prescribed reference point. Fig (\ref{helloGroup1}) shows the position of the source points, observation  points  
and reference point $P$. 
\begin{figure}[h!]
\begin{center}
 \includegraphics[scale=0.9]{./chap3.d/fig/shift_co1.eps}
\caption{ Evaluation of the potential with respect to P}
\label{helloGroup1}
\end{center}
\end{figure}
The vector  $\vec{X}$ and  $\vec{d}$ are the position vectors of the source point and observation point, respectively, with respect to  $P$.
The potential at the observation point  $\vec{r}$ is given by Eqn (\ref{sphericalGreen1}). The potential at the observation point will be evaluated with respect to $P$ using Eqn (\ref{sphericalGreen1}).
\par Using the addition theorem of vectors, we can derive the relation between position vectors $\vec{r}$, $\vec{r}_0$ and $\vec{X}$, $\vec{d}$. 
\begin{equation}
\label{sec3.5/11}
\begin{split}
\vec{r}-\vec{r}_0 = \vec{X}+\vec{d}\\
= \vec{X}-(-\vec{d})
\end{split}
\end{equation}
Substituting the aforementioned 
relation for $\vec{r}-\vec{r}_0$, the free space and spherical Green's function can be
rewritten as 
\begin{equation}
\label{sec3.4/1}
g(\vec{X},-\vec{d})= \frac{e^{ik|\vec{X}+\vec{d}|}}{4\pi|\vec{X}+\vec{d}|}\simeq \begin{cases} i\gamma\sum_{n=0}^{T}\sum_{m= -n}^{n}S_{n}^{-m}(\vec{X})R_{n}^{m}(-\vec{d}), |\vec{X}| >| \vec{d}|\\
i\gamma \sum_{n=0}^{T}\sum_{m= -n}^{n}S_{n}^{m}(-\vec{d})R_{n}^{-m}(\vec{X}), |\vec{X}| <|\vec{d}|
\end{cases}
\end{equation}
We will focus on the case when $|\vec{X}| <|\vec{d}|$. Eqn (\ref{sec3.4/1}) can be expressed for large number of source points. 
Fig (\ref{centerSource})  shows the  sources $s_{1},s_{2},...,s_{N}$ with position vectors  $X_{1},X_{2},...,X_N$. As we can see,
the reference point  $P$ is geometrically located at the cluster center of the source points.
\begin{figure}[h!]
\begin{center}
\includegraphics[scale=0.7]{./chap3.d/fig/chap3_fig3.2.eps}
\caption{ the source locations with respect to reference point}
\label{centerSource}
\end{center}
\end{figure}
The potential at the observation point $\vec{r}$  when 
$|\vec{d}|$ $>$ $|\vec{X}|$ is given by
\begin{equation}
\label{sec3.5/constantMake}
  p(\vec{d}) = i\gamma\sum_{j=0}^{N}s_{j} w_j \sum_{n=0}^{T}\sum_{m= -n}^{n}R_{n}^{-m}(\vec{X_{j}}) S_{n}^{m}(-\vec{d})
 \end{equation}
The summation in the index $j$ can be equated to the coefficient 
 \begin{equation}
\label{sec3.5/constant1}
 a_{nm} = \sum_{j=0}^{N} R_{n}^{-m}(\vec{X_{j}}) s_j w_j
 \end{equation}
 Substituting the above expression in Eqn  (\ref{sec3.5/constantMake}) yields
 \begin{equation}
\label{sec3.5/constant2}
 p(\vec{d}) = i\gamma\sum_{n=0}^{T}\sum_{m= -n}^{n} a_{nm} S_{n}^{m}(-\vec{d})
 \end{equation}
where $N$ is the total number of 
sources close to the reference point $P$.

\newpage

\section{Numerical Results for  the Accuracy of the Potential with the Expansion Coefficients}
\subsection{Accuracy for the Green's function}
In this section, we will compare the free space Green's function  $g_f(\vec{r},\vec{X})$ to the spherical Green's function $g_s(\vec{r},\vec{X})$ which is calculated using  expansion coefficients. 

\begin{equation}
\label{previousFreeSpG}
 g_f(\vec{r},\vec{X})=   \frac{e^{i\gamma|\vec{r}-\vec{X}|}}{4\pi|\vec{r}-\vec{X}|}  
\end{equation}   
\begin{equation}
\label{previousSphG}
  g_s(\vec{r},\vec{X}) =  i\gamma \sum_{n=0}^{T}\sum_{m= -n}^{n}R_{n}^{-m}(\vec{X}) S_{n}^{m}(\vec{r})
 \end{equation}
The development of the problem is as follows. The observation grid is considered as $ 1 \le x \le 41$, $1 \le y \le 41$, $z = 0$ and the source is  located on  $(10,10,0)$. Total number of grid points are $100$ including the source point.   Green's function is calculated for all the points with distance more than the distance  of the source point from the origin of the coordinate system.   In this case, the function is calculated for the all the points with distance more than $\sqrt{10^2+10^2}=14.14$ and  $\gamma=1$.  For the Eqn (\ref{previousSphG}) truncation order chosen is $47$. 
\par 
Contour plots are used to demonstrate the numerical results as shown  in Fig (\ref{cntrparam121111}). Lines show the magnitude of $g_f(\vec{r},\vec{X})$  and points show the magnitude of $g_s(\vec{r},\vec{X})$. Ideally, each set of points should follow the corresponding lines in the Fig (\ref{cntrparam121111}). The largest error occurs at $(10,10,0)$ as can be seen in Fig (\ref{cntrparam121111}).
The results show that free space and spherical Green's function are approximately equal.
\newpage
\begin{figure}[h!]
\begin{center}
\includegraphics[scale=0.9]{./chap3.d/fig/cntrParam.eps}
\caption{Error Analysis for the Green's function }
\label{cntrparam121111}
\end{center}
\end{figure}



\newpage 
\subsection{Approximation Error for the Potential with the Expansion Coefficients}
Consider comparing the potential  calculated with the  free space Green's function $p_f(\vec{r})$ to the potential calculated with the expansion coefficients  $p_s(\vec{r})$  for the multipole source case.  The potential at the observation point using  free space Green's function is given by 
\begin{equation}
\label{previousFreeSpPot}
 p_f(\vec{r})=  \sum_{i=0}^{N}s_{i} \frac{e^{i\gamma|\vec{r}-\vec{X}_i|}}{4\pi|\vec{r}-\vec{X}_i|}  
\end{equation}   
The potential at the observation point using the expansion coefficients is given by (using Eqns (\ref{sec3.5/constantMake} to \ref{sec3.5/constant2}))
\begin{equation}
\label{previousSphPot}
  p_s(\vec{r}) =  i\gamma \sum_{n=0}^{T}\sum_{m= -n}^{n}a_{nm} S_{n}^{m}(\vec{r})
 \end{equation}
where   
\begin{equation}
  a_{nm} = \sum_{i=j}^{N}s_{j}R_{n}^{-m}(\vec{X}_j)
 \end{equation}
For the numerical results, we will use  Eqns (\ref{previousFreeSpPot} and \ref{previousSphPot}).  The observation grid is considered as 
 $ 1 \le x \le 41$, $1 \le y \le 41$ and $z = 0$. The total number of points chosen are $100$ including source points. Four source points are chosen, location of the each source is $s_1=(10,10,0)$, $s_2=(-5,-5,0)$, $s_3=(2,2,0)$, $s_4=(-2,-2,0)$ and the magnitude of the each source is $1$. To calculate $p_s(\vec{r})$ the  truncation order chosen is $35$.  First the coefficients are calculated for each source and these coefficients are used to calculated the potential $ p_s(\vec{r})$  using Eqn (\ref{previousSphPot}).  In this case, the potential is calculated for the all the points with distance more than $\sqrt{10^2+10^2}=14.14$ and  $\gamma=1$. 
\par
The results are plotted in Fig (\ref{cntrparam1Aprl7}) and  contour plots are used to  demonstrate the  results. In  Fig (\ref{cntrparam1Aprl7}), lines show the magnitude of $p_f(\vec{r})$  and points show the magnitude of $p_s(\vec{r})$ . Ideally, each set of points should follow the corresponding lines in the Fig (\ref{cntrparam1Aprl7}). Largest error occurs at $(10,10,0)$ as can be seen in Fig (\ref{cntrparam1Aprl7}). Error decreases with the increasing distance from the source point. Results show that potential calculated with the expansion coefficients is a good approximation.





\begin{figure}[h!]
\begin{center}
\includegraphics[scale=0.9]{./chap3.d/fig/ErrorPlotAprl7.ps}
\caption{ Error analysis for  the potential} 
\label{cntrparam1Aprl7}
\end{center}
\end{figure}